\section{تراکنش‌ها}
یکی از زیبایی‌های طراحی که انجام دادیم این است که نیازی به تراکنش در این سیستم با خواسته‌های
فعلی وجود ندارد! به عنوان مثال در زمان ثبت نام در درسی تنها
\lr{write}
در دیتابیس فقط در دیتابیس خود
\lr{enrollment service}
اتفاق می‌افتد. در قبل از آن صرفا یک سری درخواست به سرویس‌های دیگر داده می‌شود و چک می‌شود که آیا
کاربر شرایط اخذ درس را دارد یا خیر. اما نکته‌ای که وجود دارد این است که مثلا در همین سناریو انتخاب
واحد با این که نیازی به تراکنش نداریم ولی نیاز به
\lr{lock}
بر روی کاربر داریم. به عنوان مثال در صورتی که دو درس را به صورت همزمان پردازش کنیم ممکن
است که سقف واحد دانشجویی رد شود.
(مشکل \lr{race} رخ دهد.)

یا مثلا در سناریو ثبت نمره لازم نیست که نمره‌های بقیه‌ی دانشجویان را پاک کنیم اگر ثبت نمره‌ی یک نفر
به مشکل خورد. می‌توان آن فرد را صرفا برایش یک
\lr{incident}
ثبت کرد و بعدا آن فرد را به صورت جداگانه دوباره برایش نمره را ثبت کرد.