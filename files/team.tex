\section{تیم}
به نظر من بهتر است که به خاطر گسستگی که تمامی سرویس‌ها از هم دارند ساختار تیمی افقی داشته باشیم.
از نظر من بهتر است که تیم‌هایی مانند زیر داشته باشیم:
\begin{enumerate}
    \item \textbf{تیم پنل ادمین}: این تیم برنامه نویسی سرویس‌هایی همچون \lr{staff} و \lr{professor} را بر عهده دارد. از دیتابیس گرفته تا فرانت‌اند همگی دست این تیم هستند.
    \item \textbf{تیم دانشجو}: این تیم صرفا سرویس دانشجو را مدیریت می‌کند.
    \item \textbf{تیم انتخاب واحد}: این تیم سرویس‌های \lr{enrollment} و \lr{enrollment websocket} را مدیریت می‌کند و یک \lr{frontend} متفاوت برای سایت انتخاب واحد بالا می‌آورد (دقیقا مثل \lr{myedu}).
    \item \textbf{تیم امنیت}:‌ این تیم سرویس \lr{authorization} و مدیریت امنیت کلی سایت را بر عهده دارند. دقت کنید که این تیم به صورت تکی \lr{frontend} نمی‌زند ولی حتما باید مدیریت کنند که کدی که فرانت‌اند تمامی تیم‌های دیگر می‌زنند باگ امنیتی نداشته باشند.
\end{enumerate}