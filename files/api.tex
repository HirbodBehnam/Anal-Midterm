\section{API}
\subsection{درخواست‌های فرانت‌اند}
در این قسمت به بررسی
\lr{API}های
تمامی سرویس‌ها می‌پردازیم که کاربران
(یعنی اساتید و دانشجویان و مسئولان آموزشی)
و به صورت کلی فرانت‌اند می‌تواند با آن‌ها در تعامل باشد. اولین قسمت مسیر هر
\lr{URL}
نشان‌دهنده‌ی سرویسی است که این درخواست به آن می‌رود.
\begin{latin}
\begin{lstlisting}
POST /auth/login
Body Fields:
    - username: string
    - password: string
\end{lstlisting}
\end{latin}
\begin{latin}
\begin{lstlisting}
POST /staff/add-course
Body Fields:
    - id: int
    - department-id: int
    - name: string
    - requirements: [int]
\end{lstlisting}
\end{latin}
\begin{latin}
\begin{lstlisting}
POST /professor/add-lectured-course
Body Fields:
    - id: int
    - days: [datetime]
    - final-exam: datetime
    - capacity: int
\end{lstlisting}
\end{latin}
\begin{latin}
\begin{lstlisting}
POST /professor/submit-scores
Body Fields:
    - course-id: int
    - course-group: int
    - semester: int
    - grades: map[int, float]
\end{lstlisting}
\end{latin}
\begin{latin}
\begin{lstlisting}
{POST,PUT,DELETE} /enrollment/enroll
Body Fields:
    - course-id: int
    - course-group: int
\end{lstlisting}
\end{latin}
\begin{latin}
\begin{lstlisting}
GET /enrollment/enroll
\end{lstlisting}
\end{latin}
\begin{latin}
\begin{lstlisting}
GET /enrollment/courses
Query Parameters:
    - department: int
\end{lstlisting}
\end{latin}
\begin{latin}
\begin{lstlisting}
GET /student/get-grades
\end{lstlisting}
\end{latin}
\subsection{درخواست‌های داخلی}
بعضا برخی از سرویس‌ها نیاز دارن که به صورت
\lr{synchronous}
یک سرویس دیگر را فراخوانی کنند. در این حالت حتما باید به صورت مستقیم به یکی از سرویس‌ها درخواست
زده بشود که عملا یک درخواست داخلی حساب می‌شود. در اینجا من پیشنهاد می‌کنم که حتما از
\lr{gRPC}
استفاده کنیم تا حداکثر سرعت و کمترین پهنای باند را بین میکروسرویس‌ها بگیریم.
به عنوان مثال ما در
\lr{student service}
نیاز داریم که
\lr{API}ای
داشته باشیم که بتوان به کمک آن مشخصات مربوط به انتخاب واحد دانشجو مثل حداکثر تعداد واحد و
وضعیت مجاز بودن و زمان انتخاب واحد را گرفت. همچنین مثلا نیاز به
\lr{API}ای
داریم که به کمک آن بتوان چک کرد که آیا شخصی درسی را در ترم‌های قبل برداشته است یا خیر.
یا مثلا در
\lr{staff}
نیاز داریم که لیست تمامی درس‌های دانشکده‌های مختلف را بگیریم و در
\lr{professor}
نیز نیاز داریم که لیست تمامی درس‌های استادی در ترم خاصی را بگیریم.