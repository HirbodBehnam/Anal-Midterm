\section{API}
\subsection{درخواست‌های فرانت‌اند}
در این قسمت به بررسی
\lr{API}های
تمامی سرویس‌ها می‌پردازیم که کاربران
(یعنی اساتید و دانشجویان و مسئولان آموزشی)
و به صورت کلی فرانت‌اند می‌تواند با آن‌ها در تعامل باشد. اولین قسمت مسیر هر
\lr{URL}
نشان‌دهنده‌ی سرویسی است که این درخواست به آن می‌رود. همچنین دقت کنید که در همه‌ی
\lr{API}ها
غیر از
\verb|login|
نیاز است که توکن ما در هدر‌های
\lr{HTTP}
به مقصد فرستاده شود. به کمک همین موضوع می‌توان کاربر فرستنده را اهراز هویت کرد. حال
\lr{API}های
مورد نظر را شرح می‌دهیم:

در ابتدا این
\lr{endpoint}
کار که می‌کند این است که یک نام‌کاربری و رمزعبور را صحت سنجی می‌کند و به کسی که درخواست زده است
بر می‌گرداند که کاربر مورد نظر دانشجو یا استاد یا مسئول آموزش است و یااینکه اصلا رمز عبور
کاربر درست است یا خیر. همچنین در صورتی که رمز عبور درست بود یک
\lr{session token}
نیز برگردانده می‌شود.
\begin{latin}
\begin{lstlisting}
POST /auth/login
Body Fields:
    - username: string
    - password: string
\end{lstlisting}
\end{latin}
در ادامه باید یک
\lr{API}
تعریف کنیم که مسئولان آموزش به کمک آن بتوانند که یک درس جدید تعریف بکنند. در
\lr{syntax}
زیر
\verb|[int]|
به معنی یک آرایه از
\verb|int|
است. برای تعریف درس به نام درس، شماره درس، دپارتمانی که آن درس به آن تعلق دارد و پیش‌نیاز‌های آن نیاز است.
\begin{latin}
\begin{lstlisting}
POST /staff/add-course
Body Fields:
    - id: int
    - department-id: int
    - name: string
    - requirements: [int]
\end{lstlisting}
\end{latin}
در ادامه بعد از تعریف درس نیاز است که
\lr{endpoint}ای
داشته باشیم که به کمک آن اساتید بتوانند اعلام آمادگی برای تدریس بکنند. در اینجا نیاز به فرمت خاصی داریم
که به کمک آن استاد بگوید که چه روز و ساعت‌هایی درس برقرار است.
(پی‌نوشت ریز: من خودم یک نمایش مشابه در \link{https://github.com/HirbodBehnam/CourseEnrollment/blob/97cf4ea1a9d3b545409a95d5578182cf76d11b82/pkg/course/class\_time.go\#L14-L22}{اینجا} قبلا درست کرده بودم)
در ادامه نیز تاریخ امتحان پایانترم و ظرفیت درس را استاد مشخص می‌کند.
\begin{latin}
\begin{lstlisting}
POST /professor/add-lectured-course
Body Fields:
    - id: int
    - days: [datetime]
    - final-exam: datetime
    - capacity: int
\end{lstlisting}
\end{latin}
همچنین اساتید باید بتوانند که نمرات یک درسی که در گذشته ارائه دادند را ثبت کنند.
برای این کار نیاز به شماره درس و گروه و ترم ارائه شده (دقت کنید که برخی اساتید سال بعد نمره می‌دهند!)
داریم. همچنین یک
\lr{dictionary}
از شماره دانشجویی به نمره نیز باید داشته باشیم که نشان دهنده‌ی نمرات دانش‌آموزان است.
به کمک این مدل استاد می‌تواند مثلا از روی یک فایل
\lr{csv}
نمرات را ثبت کند یا اینکه فردی را زودتر از بقیه نمره‌اش را ثبت کند.
\begin{latin}
\begin{lstlisting}
POST /professor/submit-scores
Body Fields:
    - course-id: int
    - course-group: int
    - semester: int
    - grades: map[int, float]
\end{lstlisting}
\end{latin}
حال به قسمت انتخاب واحد می‌رسیم. در ابتدا باید یک
\lr{API}
تعریف کنیم که امکان ثبت نام در درس را داشته باشد. همچنین با کمی فکر کردن متوجه می‌شویم که حذف
درس نیز پارامتر‌هایی مثل اخذ درس دارد. پس این دو را به صورت زیر تعریف می‌کنیم:
(دقت کنید که برای جای کمتر از دو متد HTTP را در یکجا نوشتم)
همچنین این درخواست زمانی که موفقیت آمیز است ۲۰۱ بر می‌گرداند که به دانشجو بگویید که بعدا قرار است جواب
نهایی انتخاب واحدت را دریافت کنی.
\begin{latin}
\begin{lstlisting}
{POST,DELETE} /enrollment/enroll
Body Fields:
    - course-id: int
    - course-group: int
\end{lstlisting}
\end{latin}
در ادامه نیز نیاز داریم که دانشجویان بتوانند گروه خود را عوض کنند.
\begin{latin}
\begin{lstlisting}
PUT /enrollment/enroll
Body Fields:
    - course-id: int
    - course-group: int
    - new-course-group: int
\end{lstlisting}
\end{latin}
همچنین یک
\lr{endpoint}
باید وجود داشته باشد که تمامی درس‌های ثبت نامی یک کاربر را بر می‌گرداند.
(در ترم جاری)
این درخواست از نوع
\lr{GET}
است و پارامتری ندارد.
\begin{latin}
\begin{lstlisting}
GET /enrollment/enroll
\end{lstlisting}
\end{latin}
در انتها نیز نیاز است که دانشجویان لیست درس‌های یک دپارتمان خاص را ببینند. این
\lr{API}
در سرویس انتخاب واحد تعریف شده است چرا که معمولا در انتخاب واحد به آن نیاز است و همچنین نیاز است
که تعداد ثبت نامی‌های هر درس را نیز به کاربر اعلام کنیم که در دست سرویس انتخاب واحد است.
\begin{latin}
\begin{lstlisting}
GET /enrollment/courses
Query Parameters:
    - department: int
\end{lstlisting}
\end{latin}
در آخرین
\lr{endpoint}
نیز صرفا تابعی باید تعریف کنیم که نمرات دانشجو در ترم خاصی را به او نشان بدهد.
\begin{latin}
\begin{lstlisting}
GET /student/get-grades
Query Parameters:
    - semester: int
\end{lstlisting}
\end{latin}
\subsection{درخواست‌های داخلی}
بعضا برخی از سرویس‌ها نیاز دارن که به صورت
\lr{synchronous}
یک سرویس دیگر را فراخوانی کنند. در این حالت حتما باید به صورت مستقیم به یکی از سرویس‌ها درخواست
زده بشود که عملا یک درخواست داخلی حساب می‌شود. در اینجا من پیشنهاد می‌کنم که حتما از
\lr{gRPC}
استفاده کنیم تا حداکثر سرعت و کمترین پهنای باند را بین میکروسرویس‌ها بگیریم.
به عنوان مثال ما در
\lr{student service}
نیاز داریم که
\lr{API}ای
داشته باشیم که بتوان به کمک آن مشخصات مربوط به انتخاب واحد دانشجو مثل حداکثر تعداد واحد و
وضعیت مجاز بودن و زمان انتخاب واحد را گرفت. همچنین مثلا نیاز به
\lr{API}ای
داریم که به کمک آن بتوان چک کرد که آیا شخصی درسی را در ترم‌های قبل برداشته است یا خیر.
یا مثلا در
\lr{staff}
نیاز داریم که لیست تمامی درس‌های دانشکده‌های مختلف را بگیریم و در
\lr{professor}
نیز نیاز داریم که لیست تمامی درس‌های استادی در ترم خاصی را بگیریم.