\section{معماری پیشنهادی}
در ابتدا همان طور که گفته شد در این معماری سعی می‌کنیم که تا جای ممکن از معماری
\lr{event driven}
استفاده کنیم. به عنوان مثال برای شروع اولین کاری که می‌توان کرد این است که به جای اینکه با کاوه‌نگار و پاکت
و زرینپال به صورت مستقیم در ارتباط باشیم از یک صف استفاده کنیم. در همه‌ی
\lr{usage}های
کاوه‌نگار و پاکت استفاده از صف منطقی است ولی زرینپال کمی داستانش فرق می‌کند و بعضی جا‌ها نمی‌توان از
صف استفاده کرد که در ادامه‌ی این قسمت به آن می‌پردازیم.

اما در ابتدا اجازه بدهید که سرویس‌هایی که قرار است طراحی کنیم را مشخص کنیم.
\subsection{سرویس‌ها}
در این قسمت به معرفی مختصری از هرکدام سرویس‌های مورد نیاز می‌پردازیم.
\subsubsection{Authorization}
سرویس
\lr{authorization}
همان طور که از اسمش پیدا است صرفا مدیریت سطح دسترسی کاربران را هندل می‌کند.
دیتابیسی که این سرویس در اختیار دارد صرفا کافی است که یک لیست از نام‌های کاربری
(شماره دانشجویی و اسم کاربری اساتید/مسئولان آموزش)،
 پسورد‌های آن‌ها، نوع اکانت آن‌ها
(دانشجو/مسئول آموزش/استاد)
و دپارتمان آن‌ها در اختیار دارد.

در زمانی که کاربری می‌خواهد وارد سایت بشود به صورت مستقیم به این سرویس وصل می‌شود و یک
\lr{token}
از آن می‌گیرند. من حقیقتا خیلی مشکلی نمی‌بینم که اینجا مثل
\link{https://my.edu.sharif.edu}{my.edu.sharif.edu}
از
\lr{JWT}
استفاده کنیم چون این طوری حتی لازم نیست که توکن‌هارو نگه داریم توی دیتابیس. یه ذره به صورت کلی
به نیازمندی‌های خودمون بر می‌گرده. من نگاه می‌کردم
\lr{edu}
یه توکن میده و
\lr{my.edu}
میاد
\lr{JWT}
می‌ده. همچنین علاوه بر ارتباط مستقیم برای وارد یا خارج شدن دقت کنید که سرویس‌های دیگر هم باید
از توکنی که بهشون داده می‌شود مطمئن باشند. برای همین آن‌ها نیز نیاز دارند که مستقیم به این سرویس
درخواست بزنند که توکن آن‌‌ها را چک کنند.

به صورت کلی هم به نظر من دیتابیس این سرویس از آنجا که بیشتر
\lr{read intensive}
است نیازی به
\lr{optimize}
کردن خاصی ندارد مخصوصا اگر از
\lr{JWT}
استفاده کنیم.
\subsubsection{Professor}
این سرویس عملا برای دسترسی استاد‌ها به درس‌هایشان استفاده می‌شود. دیتابیسی این سرویس 
صرفا اطلاعات عادی استاد‌ها را در اختیار دارد اما یک
\lr{API}
دارد که به اساتید اجازه می‌دهد که نمرات دانشجو‌ها را ثبت کنند. نکته‌ای که وجود دارد این است که ثبت نمره
به صورت
\lr{event driven}
است. بدین منظور که
\lr{event}ای
به محتوای
"نمره‌ی دانشجوی x در درس y گروه z برابر g"
است در یک صف فرستاده می‌شود و در ادامه سرویس
\lr{student}
آن‌را می‌خواند و نمره‌ی دانشجو را ثبت می‌کند. با این کار در فرانت ما استاد بعد از ثبت نمره درجا
اوکی را می‌گیرد و در ادامه آهسته و آهسته برای بچه‌ها نمره‌ها ثبت می‌شود.
\subsubsection{Staff}
این سرویس عملا برای مسئول آموزش دانشکده‌ها است. مهم ترین وظیفه دیتابیس این سرویس نگه‌داری
لیست درس‌هایی است که در حال ارائه شدن هستند و در گذشته نیز ارائه می‌شدند. اما علاوه بر آن این سرویس
امکان اضافه کردن یا حذف کردن درس را نیز به مسئولان آموزش می‌دهد. اضافه کردن درس ساده است و
صرفا یک ردیف جدید در دیتابیس همین سرویس ایجاد می‌شود. اما داستان در حذف درس کمی متفاوت است چرا
که باید افرادی که آن درس را داشته‌اند را نیز حذف کنیم از درس. برای این کار باید یک
\lr{event}
در صف قرار دهیم که به
\lr{Enrollment}
می‌رود و می‌گوید که تمام کسانی که فلان درس را دارند را پاک کن. سپس آن سرویس نیز باید یک پیام موفقیت آمیز
یا فیل شدن در همان صف برای سرویس
\lr{Staff}
ارسال کند که نشان دهد که که آیا آن درس از پایگاه داده‌ی
\lr{Staff}
نیز می‌تواند حذف شود یا خیر. اینجا عملا از پترن
\lr{saga}
استفاده کردیم. همچنین در سرویس
\lr{Enrollment}
نیز باید از پترن
\lr{outbox}
برای این تراکنش استفاده کنیم.
\subsubsection{Students}
در این سرویس اطلاعات دانشجویان، درس‌هایی که داشته‌اند و نمراتشان را نگه داری می‌کنیم به غیر از ترم جاری.

\subsubsection{Enrollment}
\subsubsection{Websocket Enrollment}
\subsubsection{Miscellaneous}